In many situations you want your module to rewrite some parts of the running
kernel, this include things like:
\begin{itemize}
  \item code: patching the code allow us to modify the behavior of a function;
  \item data: modifying the content of a variable, or even a pointer;
\end{itemize}
The base protection of the memory is permissions, this mechanisms provide a
set of per page flags that restrict the available operations on this page.
In order to implements certain techniques, we need to change data in
write-protected pages.

\subsection{Write Protect flag}
The first method is to modify the CPU's control register by setting the
WP flags (16th bit) to zero which disable the write protections.
With this protection disabled we can write anywhere in the memory without being
annoyed by the MPU. This approch disable the protection globally in the kernel.
\cite{wp_safe}

\begin{lstlisting}[frame=single]
#define CR0_WP 0x00010000 // Write Protect Bit (CR0:16)

static inline unsigned long read_cr0(void) {
    unsigned long val;
    asm volatile("mov %%cr0,%0\n" : "=r" (val), "=m");
    return val;
}

static inline void write_cr0(unsigned long val) {
    asm volatile("mov %0,%%cr0\n" : : "r" (val), "m");
}

static void disable_wp(unsigned long addr) {
    preempt_disable();
    barrier();
    write_cr0(read_cr0() & ~CR0_WP);
}

static void enable_wp(unsigned long addr) {
    write_cr0(read_cr0() | CR0_WP);
    barrier();
    preempt_enable();
}
\end{lstlisting}

\subsection{Page Table Entry}
\cite{kernel_memory}
This method is not specifig to the x86 architecture since we directly modify
Linux's structures that handle the pages. Unlike the previous method, we
change the permissions for a single page and not globally.
The principe is to retrieve the page of an address from the page tables and
change the entry to allow writing.

\begin{lstlisting}[frame=single]
static void pte_set_rw(unsigned long addr) {
    pte_t *pte;
    unsigned int level;

    /* retrieve the page where the address reside */
    pte = lookup_address(addr, &level);
    /* Add the write permission */
    if (pte->pte &~ _PAGE_RW)
        pte->pte |= _PAGE_RW;
}
\end{lstlisting}

\subsection{Shadow page mapping}
\cite{write_protected}
\todo{Shadow page mapping}
